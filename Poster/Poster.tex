%\title{EMC LaTeX Portrait Poster Template}
%%%%%%%%%%%%%%%%%%%%%%%%%%%%%%%%%%%%%%%%%
% a1poster Portrait Poster
% LaTeX Template
% Version 2.0 (22/06/16)
%
% The a1poster class was created by:
% Joe Rowing (JoeRowing@exeterms.ac.uk)
% 
% This template has been produced by:
% Joe Rowing at Exeter Mathematics School
%
% License:
% CC BY-NC-SA 3.0 (http://creativecommons.org/licenses/by-nc-sa/3.0/)
%
%%%%%%%%%%%%%%%%%%%%%%%%%%%%%%%%%%%%%%%%%

%----------------------------------------------------------------------------------------
%   PACKAGES AND OTHER DOCUMENT CONFIGURATIONS
%----------------------------------------------------------------------------------------

\documentclass[a1,portrait]{a1poster}

\usepackage{multicol} % This is so we can have multiple columns of text side-by-side
\columnsep=50pt % This is the amount of white space between the columns in the poster
\columnseprule=0.01pt % This is the thickness of the black line between the columns in the poster

\usepackage[svgnames]{xcolor} % Specify colors by their 'svgnames', for a full list of all colors available see here: http://www.latextemplates.com/svgnames-colors

\usepackage{times} % Use the times font


\usepackage{graphicx} % Required for including images
\graphicspath{{figures/}} % Location of the graphics files
\usepackage{booktabs} % Top and bottom rules for table
\usepackage[font=small,labelfont=bf]{caption} % Required for specifying captions to tables and figures
\usepackage{amsfonts, amsmath, amsthm, amssymb} % For math fonts, symbols and environments
\usepackage{wrapfig} % Allows wrapping text around tables and figures

\definecolor{Navy}{HTML}{002147}

\begin{document}

%----------------------------------------------------------------------------------------
%   POSTER HEADER 
% The header is divided into two boxes:
% The first is 75% wide and houses the title, subtitle, names, university/organization and contact information
% The second is 25% wide and houses the EMS Logo
% 
%----------------------------------------------------------------------------------------
\begin{center}\Huge \color{Navy} \textbf{$SEIRD$ Modelling of the Covid-19 Pandemic}\\
\color{Black}\huge\textit{How to exit a lockdown}\\[1cm]\end{center} % Title

\begin{minipage}[b]{0.75\linewidth}
 % Subtitle
\large \textbf{Harrison Mouat}\\[0.5cm] % Author(s)
\large Imperial College London\\[0.2cm] % University/organization
\texttt{harrison.mouat19@imperial.ac.uk}\\
\end{minipage}
%
\begin{minipage}[b]{0.25\linewidth}
\includegraphics[width=\linewidth]{imperial.png}\\
\end{minipage}

\vspace{.5cm} % A bit of extra whitespace between the header and poster content

%----------------------------------------------------------------------------------------

\begin{multicols}{2} % This is how many columns your poster will be broken into, by convention a portrait poster is generally split into 2 or 3 columns

%----------------------------------------------------------------------------------------
%   ABSTRACT

%An abstract is a brief summary of a research article, thesis, review, conference proceeding or any in-depth analysis of a particular subject and is often used to help the reader quickly ascertain the paper's purpose
%----------------------------------------------------------------------------------------

\color{Navy} % Colour for the abstract

\begin{abstract}

This poster uses the $SEIRD$ model, varying parameters as various lockdown measures are implemented in order to guage how the model would react were those lockdown measures to be lifted. The model also includes tourism statistics in order to show how the disease spread from the rest of the world into the UK. Specifically, we model the UK, France, Germany, Italy and the rest of the world as the 5 compartments which contain subcompartments of $SEIRD$.

\end{abstract}

%----------------------------------------------------------------------------------------
%   INTRODUCTION

%Avoid using technical definitions unless absolutely necessary. The introduction section is here to introduce your issue, so be sure to not bore your readers right away with excessive information. You can even include graphics if they will help the viewer understand the work that you have done.
%----------------------------------------------------------------------------------------

\color{DarkRed} % color for the introduction

\section*{Introduction}

Aliquam non lacus dolor, \textit{a aliquam quam} \cite{Smith:2012qr}. Cum sociis natoque penatibus et magnis dis parturient montes, nascetur ridiculus mus. Nulla in nibh mauris. Donec vel ligula nisi, a lacinia arcu. Sed mi dui, malesuada vel consectetur et, egestas porta nisi. Sed eleifend pharetra dolor, et dapibus est vulputate eu. \textbf{Integer faucibus elementum felis vitae fringilla.} In hac habitasse platea dictumst. Duis tristique rutrum nisl, nec vulputate elit porta ut. Donec sodales sollicitudin turpis sed convallis. Etiam mauris ligula, blandit adipiscing condimentum eu, dapibus pellentesque risus.

\textit{Aliquam auctor}, metus id ultrices porta, risus enim cursus sapien, quis iaculis sapien tortor sed odio. Mauris ante orci, euismod vitae tincidunt eu, porta ut neque. Aenean sapien est, viverra vel lacinia nec, venenatis eu nulla. Maecenas ut nunc nibh, et tempus libero. Aenean vitae risus ante. Pellentesque condimentum dui. Etiam sagittis purus non tellus tempor volutpat. Donec et dui non massa tristique adipiscing.

%----------------------------------------------------------------------------------------
%   CONTENTS
%----------------------------------------------------------------------------------------

\color{Black} % DarkSlateGray color for the rest of the content

\section*{Contents}

\begin{enumerate}
\item Creating the model
\item Gathering suitable data
\item Fitting the model
\item Forecasts for exiting lockdown
\item Conclusion
\end{enumerate}

\section{Creating the model}

In creating the model I took a conventional $SEIRD$ model, letting each compartment be a vector of 5 values and interlinking those values using tourism statistics between the countries, for example, the differential equation for the $\vec{S}$ vector is as follows:
\[\dfrac{d\vec{S}}{dt} = -\dfrac{\beta\vec{S}\ringdot\vec{I}}{N} + \epsilon\vec{R} + \mu N - \mu S + \vec{S}(I - E)\]
Where I is the immigration matrix and E is the emigration array, i.e. the proportion of the country travelling from one country to another. For example, $I_12$ would be the proportion of country 1 travelling to country 2, and $E_11$ is the proportion of country 1 emigrating. Also, $\ringdot$ is elementwise vector multiplication and the division by $N$ is elementwise so that the countries' models are separate apart from travel between them.

\subsection*{Mathematical Section}

Nulla vel nisl sed mauris auctor mollis non sed. 

\begin{equation}
E = mc^{2}
\label{eqn:Einstein}
\end{equation}

Curabitur mi sem, pulvinar quis aliquam rutrum. (1) edf (2)
, $\Omega=[-1,1]^3$, maecenas leo est, ornare at. $z=-1$ edf $z=1$ sed interdum felis dapibus sem. $x$ set $y$ ytruem. 
Turpis $j$ amet accumsan enim $y$-lacina; 
ref $k$-viverra nec porttitor $x$-lacina. 

Vestibulum ac diam a odio tempus congue. Vivamus id enim nisi:

\begin{eqnarray}
\cos\bar{\phi}_k Q_{j,k+1,t} + Q_{j,k+1,x}+\frac{\sin^2\bar{\phi}_k}{T\cos\bar{\phi}_k} Q_{j,k+1} &=&\nonumber\\ 
-\cos\phi_k Q_{j,k,t} + Q_{j,k,x}-\frac{\sin^2\phi_k}{T\cos\phi_k} Q_{j,k}\label{edgek}
\end{eqnarray}
and
\begin{eqnarray}
\cos\bar{\phi}_j Q_{j+1,k,t} + Q_{j+1,k,y}+\frac{\sin^2\bar{\phi}_j}{T\cos\bar{\phi}_j} Q_{j+1,k}&=&\nonumber \\
-\cos\phi_j Q_{j,k,t} + Q_{j,k,y}-\frac{\sin^2\phi_j}{T\cos\phi_j} Q_{j,k}.\label{edgej}
\end{eqnarray} 

Nulla sed arcu arcu. Duis et ante gravida orci venenatis tincidunt. Fusce vitae lacinia metus. Pellentesque habitant morbi. $\mathbf{A}\underline{\xi}=\underline{\beta}$ Vim $\underline{\xi}$ enum nidi $3(P+2)^{2}$ lacina. Id feugain $\mathbf{A}$ nun quis; magno.

%----------------------------------------------------------------------------------------
%   RESULTS 

%It’s always a good idea to begin the Results section with an initial summary of your results. Don’t address your research question just yet; instead, just address the general aspects of the data you collected or the number of valid data obtained.

%In your next paragraph, you can discuss the relationship between the data and your research question. What exactly does your data mean? Be sure to include any graphics that can help show you data visually, as the readers can understand graphics more easily and quickly than blocks of text.

%Sometimes less is more. Be selective when deciding what images, charts, and graphs make it onto your poster! 

%Charts and graphs are usually more effective than tables, but whatever you choose to use, make sure everything is labeled clearly! A graph with missing labels or a table without a title will just leave the reader confused. Also, carefully consider what type of chart or graph will best show your results. K. Broman, professor of Biostatistics & Medical Informatics at the University of Wisconsin Madison has written a great article titled, “The Top Ten Worst Graphs.” (http://www.biostat.wisc.edu/~kbroman/topten_worstgraphs/)
%----------------------------------------------------------------------------------------

\section*{Results}


Donec faucibus purus at tortor egestas eu fermentum dolor facilisis. Maecenas tempor dui eu neque fringilla rutrum. Mauris \emph{lobortis} nisl accumsan. Aenean vitae risus ante.
%
\begin{wraptable}{l}{12cm} % Left or right alignment is specified in the first bracket, the width of the table is in the second
\begin{tabular}{l l l}
\toprule
\textbf{Treatments} & \textbf{Response 1} & \textbf{Response 2}\\
\midrule
Treatment 1 & 0.0003262 & 0.562 \\
Treatment 2 & 0.0015681 & 0.910 \\
Treatment 3 & 0.0009271 & 0.296 \\
\bottomrule
\end{tabular}
\captionof{table}{\color{DarkRed} Table caption}
\end{wraptable}
%
Phasellus imperdiet, tortor vitae congue bibendum, felis enim sagittis lorem, et volutpat ante orci sagittis mi. Morbi rutrum laoreet semper. Morbi accumsan enim nec tortor consectetur non commodo nisi sollicitudin. Proin sollicitudin. Pellentesque eget orci eros. Fusce ultricies, tellus et pellentesque fringilla, ante massa luctus libero, quis tristique purus urna nec nibh.

Nulla ut porttitor enim. Suspendisse venenatis dui eget eros gravida tempor. Mauris feugiat elit et augue placerat ultrices. Morbi accumsan enim nec tortor consectetur non commodo. Pellentesque condimentum dui. Etiam sagittis purus non tellus tempor volutpat. Donec et dui non massa tristique adipiscing. Quisque vestibulum eros eu. Phasellus imperdiet, tortor vitae congue bibendum, felis enim sagittis lorem, et volutpat ante orci sagittis mi. Morbi rutrum laoreet semper. Morbi accumsan enim nec tortor consectetur non commodo nisi sollicitudin.

\begin{center}\vspace{1cm}
\includegraphics[width=0.5\linewidth]{imperial.png}
\captionof{figure}{\color{DarkRed} Figure caption}
\end{center}\vspace{1cm}

In hac habitasse platea dictumst. Etiam placerat, risus ac.

Adipiscing lectus in magna blandit:

\begin{center}\vspace{1cm}
\begin{tabular}{l l l l}
\toprule
\textbf{Treatments} & \textbf{Response 1} & \textbf{Response 2} \\
\midrule
Treatment 1 & 0.0003262 & 0.562 \\
Treatment 2 & 0.0015681 & 0.910 \\
Treatment 3 & 0.0009271 & 0.296 \\
\bottomrule
\end{tabular}
\captionof{table}{\color{DarkRed} Table caption}
\end{center}\vspace{1cm}

Vivamus sed nibh ac metus tristique tristique a vitae ante. Sed lobortis mi ut arcu fringilla et adipiscing ligula rutrum. Aenean turpis velit, placerat eget tincidunt nec, ornare in nisl. In placerat.

\begin{center}\vspace{1cm}
\includegraphics[width=0.5\linewidth]{imperial.png}
\captionof{figure}{\color{DarkRed} Figure caption}
\end{center}\vspace{1cm}

%----------------------------------------------------------------------------------------
%   CONCLUSIONS
%In your conclusion section you want to briefly review your research questions and the results you obtain. You also should add why your results are interesting or significant. TIPS: Relate your results to other published research in the field. This will give your research more impact on your readers as well as show your professionalism in the study. You can also suggest continuing research that would build upon your current study.
%----------------------------------------------------------------------------------------

\color{FireBrick} %  colour for the conclusions to make them stand out

\section*{Conclusions}

\begin{itemize}
\item Pellentesque eget orci eros. Fusce ultricies, tellus et pellentesque fringilla, ante massa luctus libero, quis tristique purus urna nec nibh. Phasellus fermentum rutrum elementum. Nam quis justo lectus.
\item Vestibulum sem ante, hendrerit a gravida ac, blandit quis magna.
\item Donec sem metus, facilisis at condimentum eget, vehicula ut massa. Morbi consequat, diam sed convallis tincidunt, arcu nunc.
\item Nunc at convallis urna. isus ante. Pellentesque condimentum dui. Etiam sagittis purus non tellus tempor volutpat. Donec et dui non massa tristique adipiscing.
\end{itemize}

\color{DarkSlateGray} % Set the color back to DarkSlateGray for the rest of the content

 %----------------------------------------------------------------------------------------
%   REFERENCES
%If you have an extremely extensive list of references, you may want to break it into 2 columns. 

% It is common to shrink the font of the References section if it becomes overbearing and long

%----------------------------------------------------------------------------------------
\begin{small}%Makes the text of the references section smaller
\begin{multicols}{2}%Makes the section two col
\nocite{*} % Print all references regardless of whether they were cited in the poster or not
\bibliographystyle{plain} % Plain referencing style
\bibliography{bibliography} % Use the example bibliography file sample.bib
\end{multicols}
\end{small}
\end{multicols}
\end{document}